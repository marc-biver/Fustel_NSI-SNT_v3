% !TeX TXS-program:compile = txs:///pdflatex/[--shell-escape]
\documentclass[11pt,a4paper]{exam}
%\printanswers % pour imprimer les réponses (corrigé)
\noprintanswers % Pour ne pas imprimer les réponses (énoncé)
\addpoints % Pour compter les points
\usepackage[utf8]{inputenc}
\usepackage{minted}

%\usepackage[margin=1 in]{geometry}
\usepackage[a4paper,top=2.5cm,bottom=2.5cm,left=1.5cm,right=1.5 cm,marginparwidth=1.75cm]{geometry}
\usepackage{amsmath,amssymb}
\usepackage{multicol}
\usepackage{graphicx}
\usepackage{setspace}
\usepackage{dashundergaps}

\newcommand{\examnum}{Contrôle majeur \#2}
\newcommand{\class}{SNT 2\textsuperscript{nde}}
\newcommand{\examdate}{15/01/2024}
\newcommand{\timelimit}{45 Minutes}
\newcommand{\lycee}{Lycée Fustel de Coulanges}

\pagestyle{head}
\firstpageheader{}{}{}
\runningheader{\class}{\examnum\ - Page \thepage\ / \numpages}{\examdate}
\runningheadrule


\begin{document}
% Espace d'en-tête
\noindent
\begin{spacing}{1}
	\noindent
	\begin{tabular*}{\textwidth}{l @{\extracolsep{\fill}} l @{\extracolsep{6pt}} l}
		\textbf{\class} & \textbf{\examnum, \examdate}&\\
		\textbf{\lycee} &\textbf{Durée: \timelimit} &\\
	\end{tabular*}\\
\end{spacing}

\noindent
\vspace{10pt}
\hrule
\vspace{5pt} 
\noindent
\\
Ce contrôle comporte \numquestions\ questions; le maximum possible de points est de \numpoints\ points.\\ 
Les réponses sont à porter sur une \uline{\textbf{copie}} (\textbf{PAS} un morceau de papier arraché d'un cahier ou une copie déchirée) \uline{comportant votre nom}.\\
Il n'est pas nécessaire de répondre aux questions dans l'ordre \textemdash\ commencez par celles où vous vous sentez le plus à l'aise (mais numérotez bien les questions sur votre copie).\\
Les calculatrices ne sont \uline{pas} autorisées.\\
\noindent
\hrule
\vspace{15pt} 

    \begin{spacing}{1,1}
        \begin{questions} % DEBUT DE L'EXAMEN
        	
        	\question QCM -- reportez sur votre copie la ou les réponses correctes \textit{en mentionnant bien la lettre de la question à chaque fois}.
			\begin{parts}
				\part[1]Le code RVB 0-255-0 correspond à la couleur:
        			\begin{choices}
		        			\choice Jaune.
		        			\choice Rouge.
		        			\correctchoice Verte.
		        			\choice Noire.
        			\end{choices}
        		\part[1]Coder une couleur sur 4 bits permet de différencier:
	        		\begin{choices}
	        			\choice 2 couleurs.
	        			\choice 4 couleurs.
	        			\choice 8 couleurs.
	        			\correctchoice 16 couleurs.
	        			\choice 32 couleurs.
	        		\end{choices}
        		\part[1]Le langage HTML sert à:
	        		\begin{choices}
	        			\correctchoice Structurer et présenter le contenu sur les pages Web.
	        			\choice Protéger les sites web contre les hackers.
	        			\correctchoice Décrire la structure de base d'un document sur Internet
	        			\choice Créer et gérer des bases de données.
	        			\choice Toutes les réponses ci-dessus sont vraies.
	        		\end{choices}
			\end{parts}
			        	
        	\question[2] Si vous désactivez les cookies dans votre navigateur, quel impact cela pourrait-il avoir sur votre navigation sur Internet ? Donnez un exemple concret (en citant un site web spécifique et au moins deux conséquences précises).
        	
        	\question[2] Un ami vous appelle en panique car il pense que son ordinateur a été infecté par un logiciel malveillant. Quels trois conseils pratiques lui donneriez-vous pour gérer la situation et protéger ses données?
        	
        	\question[2] Imaginons que la neutralité du net a été abolie: expliquez (en donnant trois exemples concrets en tout) les conséquences possibles d'une telle mesure -- pour vous, en tant qu'utilisateur d'internet, et pour une petite entreprise / startup de vente en ligne.
        	
        	\question[2] "\textit{Les moteurs de recherche coome Google, ça fonctionne parfaitement: on pose une question, ça nous donne la meilleure réponse possible, et c'est absolument tout.}" Donnez deux arguments qui pourraient aller à l'encontre d'une telle affirmation.
        	
        	\question[2] Une de vos amies, résidant en France, a trouvé lors d'une recherche Google de vieux articles négatifs la concernant publiés sur un site américain. Ces articles décrivaient des faits qui étaient vrais il y a longtemps mais qui ne le sont plus actuellement. Elle craint que cette cela nuise à sa recherche d'emploi et souhaite qu'ils soient supprimés. Elle vous demande votre avis: que lui conseillez-vous? Quels sont ses droits en termes de suppression de ces informations selon la législation, notamment le RGPD?
        	
        	\question Considérez le code HTML suivant:
        	\begin{verbatim}
<!DOCTYPE html>
<html lang="fr">
    <head>
        <title>Ma page HTML
    </head>
    <body>
        <h1>Mes gouts</h1>
        <h2>La lecture</h2>
        <img src="livre.jpg">
        <br /> <br /> <br />
        Ce que j'aime lire:
        <ul>
            <li>Des livres</a></li>
            <li><a www.lemonde.fr>Le journal</a></li>
        <FinDeListe>
        <h2>Le sport</h2>
    </body>
</html>
        	\end{verbatim}
        	\begin{parts}
        		\part[2] Ce code contient plusieurs erreurs: identifiez-en au moins deux et expliquez comment il faut les corriger.
        		\part[1] En supposant que le code a été corrigé, dessinez sur votre copie le rendu qu'aurait cette page dans un navigateur web comme Firefox par exemple.
        		\part[1] En supposant toujours que le code a été corrigé, où s'afficherait le texte "Ma page HTML"?
        	\end{parts}
        		
			\question[2 \half] Considérez le texte suivant:
			
			\textit{Dans un appareil photo numérique, la lumière passe à travers \uline{\ \ \ \ }\textbf{(A)}\uline{\ \ \ \ } et frappe le \uline{\ \ \ \ }\textbf{(B)}\uline{\ \ \ \ }, qui est composé de millions de \uline{\ \ \ \ }\textbf{(C)}\uline{\ \ \ \ } sensibles à la lumière; chacun de ces \uline{\ \ \ \ }\textbf{(C)}\uline{\ \ \ \ } convertit la lumière en un signal analogique, puis un signal numérique dans lequel l'image est découpée en \uline{\ \ \ \ }\textbf{(D)}\uline{\ \ \ \ }, avant d'être stockée dans la \uline{\ \ \ \ }\textbf{(E)}\uline{\ \ \ \ } de l'appareil pour une visualisation ultérieure.}
			
			Donnez les mots manquants dans ce texte -- de \textbf{(A)} à \textbf{(E)}.
			
			\question On considère une image numérique de forme carrée de 1000 pixels de côté, que l'on imprime sur une feuille 10 pouces (environ 25 cm) de côté.
			
			\begin{parts}
				\part[1] Quelle est la définition de cette image?
				\part[1] Sa résolution?
				\part[\half] Combien devrait mesurer le côté de la feuille pour atteindre une résolution de 400 ppp?				
			\end{parts}
  
        \end{questions}
    \end{spacing}
\end{document}