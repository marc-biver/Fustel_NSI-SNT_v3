% !TeX TXS-program:compile = txs:///pdflatex/[--shell-escape]
% Le truc au-dessus pour avoir l'option shell-escape qui permet de faire du minted.
\documentclass[12pt]{article}

\newcommand{\thmquiz}{un peu de tout...}
\newcommand{\idquiz}{04}

\newcommand{\classe}{1\textsuperscript{ère} NSI}

% Affichage ou non des reponses aux questions & exercices
\newif\ifDispRep
%\DispReptrue  % Show the text
\DispRepfalse % Hide the text

% Incorporation tous éléments de préambule communs à tous mes cours
% Fonctionne comme ceci quand la variable d'environnement TEXINPUTS est bien positionnée ET prise en compte - dans mon cas ça implique de lancer TexStudio par script interposé. (voir Evernote)
% Sinon on donne le chemin complet:
%\usepackage{/Users/marcbiver/Documents/01_Prof/10_Forge_AEIF/_Forge/CoursLFC}
\usepackage{CoursLFC}



\begin{document}
	\thispagestyle{empty} % pas de header / footer - il ramenait une ligne dasn l'en-tête qui devait venir de quelque part dans CoursLFC
	% Titre minimaliste
	\begin{center}
		\setlength{\fboxrule}{2pt}
		\setlength{\fboxsep}{5pt}
		\fbox{\textbf{{\large \classe \ --- Quiz \idquiz: \thmquiz}}}
		\vspace{0.5em}
	\end{center}
	
	\begin{MonQz}{Parcours de dictionnaire}
		\MonPython{04_2.py}
		Compléter (A, B, C, et D) le code de la fonction \texttt{SujetsFaciles} pour qu'elle fasse ce qui est annoncé.
	\end{MonQz}
	\begin{MaReponse}
	\end{MaReponse}
	
	\begin{MonQz}{Invariant de boucle}
		\MonPython{04_1.py}
		Cette fonction est censée calculer la factorielle de n (notée $n!$ qui vaut $n \times (n-1) \times \text{(...)} \times 2$). Démontrer que c'est vrai en utilisant un invariant de boucle.
	\end{MonQz}
	\begin{MaReponse}
	\end{MaReponse}
	
%	\begin{MonQz}{Codage d'un entier naturel}
%		\begin{alphenum}
%			\item Quel est le nombre maximal que je peux coder sur 5 bits?
%			\item Combien de bits me faudra-t-il au minimum pour coder le nombre 100?
%		\end{alphenum}
%	\end{MonQz}
%	\begin{MaReponse}
%	\end{MaReponse}
	
	\begin{MonQz}{Code mystère}
		\MonPython{04_4.py}
		Que va afficher ce code? (note: cette syntaxe de print permet d'afficher du texte bout à bout sans passage à la ligne. Ainsi "\texttt{print('a', end=""); print('b', end="")})" affichera "\texttt{ab}").
		
	\end{MonQz}
	\begin{MaReponse}
	\end{MaReponse}
	
	\begin{MonQz}{Message d'erreur}
		Zut -- j'ai la bonne réponse à la question 1 mais quand je teste ma fonction \texttt{SujetsFaciles} j'ai ce message d'erreur: que veut-il dire?
		\begin{verbatim}
			SujetsFaciles(2, LstSujets)
			File "C:\Users\Marc\PyProj\exemple.py", line 11, in SujetsFaciles
			if lst[i]['difficulte'] < seuil:
			TypeError: '<' not supported between instances of 'str' and 'int'
		\end{verbatim}
	\end{MonQz}
	\begin{MaReponse}
	\end{MaReponse}

\end{document}