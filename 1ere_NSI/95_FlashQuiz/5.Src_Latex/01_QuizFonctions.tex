% !TeX TXS-program:compile = txs:///pdflatex/[--shell-escape]
% Le truc au-dessus pour avoir l'option shell-escape qui permet de faire du minted.
\documentclass[12pt]{article}

\newcommand{\thmquiz}{fonctions}
\newcommand{\idquiz}{01}

\newcommand{\classe}{1\textsuperscript{ère} NSI}

% Affichage ou non des reponses aux questions & exercices
\newif\ifDispRep
\DispReptrue  % Show the text
%\DispRepfalse % Hide the text

% Incorporation tous éléments de préambule communs à tous mes cours
%\usepackage{../../../CoursLFC}
% Windows / Atos janvier 2024 - passé à variable d'environnement TEXINPUTS (voir Evernote)
\usepackage{CoursLFC}
% VRAI PROBLEME: POURQUOI IL NE PREND PAS LE TEXINPUTS EN ENTREE et que du coup j'ai besoin de donner le chemin absolu....?
% Finalement ça marche quand je lance TexStudio depuis le terminal alors c'est ce que je fais... (voir, toujours, Evernote). Soit.
%\usepackage{/Users/marcbiver/Documents/01_Prof/10_Forge_AEIF/_Forge/CoursLFC}


\begin{document}
	\thispagestyle{empty} % pas de header / footer - il ramenait une ligne dasn l'en-tête qui devait venir de quelque part dans CoursLFC
	% Titre minimaliste
	\begin{center}
		\setlength{\fboxrule}{2pt}
		\setlength{\fboxsep}{5pt}
		\fbox{\textbf{{\large \classe \ --- Quiz \idquiz: \thmquiz}}}
		\vspace{0.5em}
	\end{center}
	
	\begin{MonQz}{retour de fonction}
		\MonPython{01_1.py}
		Que retourne cette fonction?
		\begin{alphenum}
			\item La somme de a et b
			\item Rien
			\item Une erreur
			\item Le produit de a et b
		\end{alphenum}
	\end{MonQz}
	\begin{MaReponse}
		Réponse \textbf{b} -- en effet cette fonction, de fait, ne fait rien puisqu'elle n'affiche rien à l'utilisateur (pas de \texttt{print}) et ne renvoie aucune valeur qui pourrait être utilisée dans le programme qui l'a appellée (pas de \texttt{return}).
	\end{MaReponse}
	
	\begin{MonQz}{retour de fonction}
		\MonPython{01_2.py}
		Si j'execute ce code, qu'est-ce qui va s'afficher à l'écran?
	\end{MonQz}
	\begin{MaReponse}
		L'affichage sera:
		\begin{verbatim}
			Calcul en cours...
		\end{verbatim}
		... et strictement rien d'autre! Un \texttt{return} n'affiche rien et, surtout, un \texttt{return} force une sortie de la fonction; donc "\texttt{Calcul termine!}" ne pourra jamais s'afficher.
	\end{MaReponse}
	
	\begin{MonQz}{retour de fonction}
		\MonPython{01_3.py}
		Que vaut la variable \texttt{res}?
	\end{MonQz}
	\begin{MaReponse}
		Elle ne vaut rien!! (ou, plus exactement, elle vaut \texttt{None}) Un \texttt{print} n'est \textit{\textbf{pas}} un retour, c'est juste un affichage -- donc en l'état \texttt{division(a, b)} ne renvoie strictement rien.
	\end{MaReponse}

	\begin{MonQz}{message d'erreur}
		Que veut dire ceci?
		\begin{verbatim}
			Traceback (most recent call last):
			File "C:\Users\Marc\PyProj\exemple.py", line 4, in <module>
			MaFonction()
			TypeError: MaFonction() missing 1 required positional argument: 'a'
		\end{verbatim}
	\end{MonQz}
	\begin{MaReponse}
		Le message d'erreur est très clair: "missing 1 required positional argument: 'a'". L'appel à une fonction a été effectué avec un argument manquant -- et le message précise même le nom du paramètre correspondant, 'a'. Et à la ligne précédente on voit l'appel fautif en question:
		\begin{verbatim}
			MaFonction()
		\end{verbatim}
		Donc on peut même (presque) en déduire la syntaxe exacte de la ligne de définition de la fonction:
		\begin{verbatim}
			def MaFonction(a):
		\end{verbatim}
	\end{MaReponse}

\end{document}