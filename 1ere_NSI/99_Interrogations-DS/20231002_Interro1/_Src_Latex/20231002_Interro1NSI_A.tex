% !TeX document-id = {58e5ef0b-ac0e-469b-bfec-b82861a95ae8}
% !TeX TXS-program:compile = txs:///pdflatex/[--shell-escape]

% NOTE: 2B = même chose, juste l'ordre des séquences de code modifié pour éviter la triche

\documentclass[11pt,a4paper]{exam}
\addpoints % Pour compter les points
\usepackage[utf8]{inputenc}
\usepackage{minted}

\usepackage{geometry}
\usepackage{amsmath,amssymb}
\usepackage{multicol}
\usepackage{graphicx}
\usepackage{setspace}
\usepackage{dashundergaps}
%\usepackage[monochrome]{xcolor} % Permet de tout mettre en N&B strict
\usepackage{xcolor}
\selectcolormodel{gray} % Permet de tout mettre en niveaux de gris
\geometry{left=1.5cm, right=1.5cm, top=2.2cm, bottom=2.2cm} % Définition des marges du doc

\newcommand{\class}{1\textsuperscript{ère} Spé. NSI Gr. 2\textsubscript{A}}
\newcommand{\examnum}{Interrogation \#1}
\newcommand{\examdate}{02/10/2023}
\newcommand{\timelimit}{20 Minutes}
\newcommand{\lycee}{Lycée Fustel de Coulanges}

\pagestyle{head}
\firstpageheader{}{}{}
\runningheader{\class}{\examnum\ - Page \thepage\ / \numpages}{\examdate}
\runningheadrule


\begin{document}
% Espace d'en-tête
    \noindent
    \begin{spacing}{2}
        \noindent
        \begin{tabular*}{\textwidth}{l @{\extracolsep{\fill}} l @{\extracolsep{6pt}} l}
            \textbf{\class} & \textbf{\examnum, \examdate}&\\
            \textbf{\lycee} &\textbf{Durée: \timelimit} &\\
        \end{tabular*}\\
    \end{spacing}

    \noindent
    \hrule
    \vspace{5pt} 
    \noindent
    \\
    Cette interrogation comporte \numquestions\ questions; elle sera notée sur \numpoints\ points. 
    Les réponses sont à porter sur une copie \uline{comportant votre nom}.\\
    \noindent
    \hrule
    \vspace{10pt} 

    \begin{questions} % DEBUT DE L'EXAMEN
        \begin{spacing}{1,2}
    
            % #1 - tableau - OK
            \question[14] Correction de code: dans cet exercice il vous est demandé, dans des extraits 
            de code Python où \uline{les lignes sont numérotées} et \uline{les blocs identifiés par des lettres}, de trouver les erreurs 
            et d'en proposer une correction en précisant le numéro de ligne.
            \\
            Exemple: si le code suivant vous est proposé:
            \begin{minted}{Python}
                01: # A: Affichage message d'accueil
                02: print("Bonjour!"
            \end{minted}
        	Vous répondrez: \hspace{1cm}\textit{Bloc A, ligne 2, il faut refermer la parenthèse: print("Bonjour!")}
            \leavevmode
            \\
            \\
            % Code à évaluer pour l'exercice
            \noindent
            \begin{minipage}{0.5\textwidth}
                \begin{minted}{Python}
01: # A: Calcul d'une somme
02: a = 5
03: b = a + c
04: print(b)

05: # B: Fonc. pour afficher un texte
06: def fonction_aff(txt):
07: print("Ceci est le texte: ", txt)

08: # C: Affichage message d'accueil
09: print("Bonjour!')

10: # D: Calcul d'une valeur au carre
11: x = "15"
12: resultat = x ** 2
13: print(resultat)

14: # E: Fonc. qui double un nombre
15: def CalculDouble(nombre):
16:     resultat = nombre * 2
17:
18: a = 10
19: print(CalculDouble(a))
                \end{minted}
            \end{minipage}
            \begin{minipage}{0.5\textwidth}
                \begin{minted}{Python}
20: # F: Fonc. qui trouve le max
21: def MyMax(a, b):
22: if a > b:
23:     return a
24: else:
25:     return b

26: # G: Affich. produit
27: a = 27
28: b = 12
29: c = a * b
30: print("Le produit vaut ", c)

31: # H: Evaluation mineur / majeur
32: age = input("Quel est votre age?")
33: if age < 18:
34:     print("Vous etes mineur.")
35: else:
36:     print("Vous etes majeur.")

37: # I: Verification d'egalite
38: x = 5
39: y = 10
40: if x = y:
41:     print("x est egal a y")
                \end{minted}
            \end{minipage}
            \leavevmode
            \\
            \\
            \\
            Dans les 9 blocs précédents (A \textemdash\ I), 7 comportent une erreur: trouvez-les, et corrigez-les comme indiqué ci-dessus.
            \newpage

            % Page 2
            \leavevmode
            \\
            % Question 2
            \question Considérez la fonction suivante:
            \begin{minted}{Python}
def eligibilite(nom, age):
    if (nom == "Alice" or nom == "Bob") and age > 18:
        return "VRAI"
    else:
        return "FAUX"
            \end{minted}
            Qu'est-ce qui apparaitra à l'écran à l'issue des appels suivants?
            \begin{parts}
                \part[\half]
                    \texttt{print(eligibilite("Alice", 18))}
                \part[\half]
                    \texttt{print(eligibilite("Eve", "Bob"))}	
                \part[\half]
                    \texttt{print(eligibilite("Bob", 20))}
                \part[\half]
                    \texttt{print(eligibilite("Thomas", 23))}
            \end{parts}
%        \end{spacing}
        
        
        % Question 3 - HTML
        \question[4] Considérez le code HTML suivant:
%        \begin{spacing}{1}
            \begin{minted}{HTML}
<!DOCTYPE html>
<html lang="en">
<head>
<meta charset="UTF-8">
<link rel="stylesheet" href="./css/style.css">
<title>Ma page HTML
</head>
<body>
    Voici une liste:
    <ul>
        <li>1er element</a></li>
        <li><a href="www.google.com">SURPRISE</a></li>
    </ul>


    <img src="www.amazon.com">

</body>
</html>
            \end{minted}
%        \end{spacing}
%        \begin{spacing}{1,2}
            Ecrivez une critique de ce code sur la base des pratiques qu'on a apprises en 
            cours \textemdash\ indice: votre critique devrait soulever \uline{\textit{au moins}} 
            4 éléments à corriger.
            \\
            Question bonus: si vous ouvrez cette page HTML dans un navigateur, qu'est-ce 
            qui va s'afficher selon vous?
        \end{spacing}
    \end{questions}
\end{document}