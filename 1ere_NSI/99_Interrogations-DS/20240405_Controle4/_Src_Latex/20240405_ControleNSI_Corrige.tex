% !TeX document-id = {63e2cb4c-39e5-4fe5-ae31-d450d2286238}
% !TeX TXS-program:compile = txs:///pdflatex/[--shell-escape]
\documentclass[11pt,a4paper]{exam}
%\printanswers % pour imprimer les réponses (corrigé)
\noprintanswers % Pour ne pas imprimer les réponses (énoncé)
\addpoints % Pour compter les points
\usepackage[utf8]{inputenc}
\usepackage{minted}

\usepackage{algorithm} % pour faire du pseudo-code
\usepackage{algorithmicx} % pour faire du pseudo-code
\usepackage{algpseudocode} % pour faire du pseudo-code

% Traduction des commandes en pseudo-code
\renewcommand{\algorithmicfor}{\textbf{pour}}
\renewcommand{\algorithmicif}{\textbf{si}}
\renewcommand{\algorithmicthen}{\textbf{alors}}
\renewcommand{\algorithmicelse}{\textbf{sinon}}
\renewcommand{\algorithmicfunction}{\textbf{fonction}}
\renewcommand{\algorithmicforall}{\textbf{pour tout}}
\renewcommand{\algorithmicdo}{\textbf{faire}}
\renewcommand{\algorithmicwhile}{\textbf{tant que}}
\renewcommand{\algorithmicend}{\textbf{fin}}
\renewcommand{\algorithmicreturn}{\textbf{retourner}}
\renewcommand{\algorithmicrequire}{\textbf{Entrée:}}
\renewcommand{\algorithmicensure}{\textbf{Sortie:}} % Détournement du Ensure, mais ce n'est pas très grave...

\usepackage{geometry}
\usepackage{amsmath,amssymb}
\usepackage{multicol}
\usepackage{graphicx}
\usepackage{setspace}
\usepackage{dashundergaps}
%\usepackage[monochrome]{xcolor} % Permet de tout mettre en N&B strict
\usepackage{xcolor}
%\selectcolormodel{gray} % Permet de tout mettre en niveaux de gris
\geometry{left=0.7cm, right=0.7cm, top=1.5cm, bottom=1cm} % Définition des marges du doc

%\newcommand{\class}{1\textsuperscript{ère} Spé. NSI Gr. 2\textsubscript{A}}
\newcommand{\class}{1\textsuperscript{ère} Spé. NSI Gr. 2}
\newcommand{\examnum}{Contrôle \#4}
\newcommand{\examdate}{05/04/2024}
\newcommand{\timelimit}{1 Heure}
\newcommand{\lycee}{Lycée Fustel de Coulanges}

\pagestyle{head}
\firstpageheader{}{}{}
\runningheader{\class}{\examnum\ - Page \thepage\ / \numpages}{\examdate}
\runningheadrule


\begin{document}
% Espace d'en-tête
    \noindent
    \begin{spacing}{1}
        \noindent
        \begin{tabular*}{\textwidth}{l @{\extracolsep{\fill}} l @{\extracolsep{6pt}} l}
            \textbf{\class} & \textbf{\examnum, \examdate}&\\
            \textbf{\lycee} &\textbf{Durée: \timelimit} &\\
        \end{tabular*}\\
    \end{spacing}

    \noindent
    \vspace{10pt}
    \hrule
    \vspace{5pt} 
    \noindent
    \\
    Ce contrôle comporte \numquestions\ questions; le nombre maximal possible de points est de \numpoints. 
    Les réponses sont à porter sur une copie \uline{comportant votre nom}.
    Il n'est pas nécessaire de répondre aux questions dans l'ordre \textemdash\ commencez par celles où vous vous sentez le plus à l'aise \textit{(mais ne tentez les questions bonus qu'après avoir fini le reste!!)}.\\
    Les calculatrices ne sont \uline{pas} autorisées.\\
    \noindent
    \hrule
    \vspace{15pt} 

    \begin{questions} % DEBUT DE L'EXAMEN
		\begin{spacing}{1}
    
		\question \textit{Questions de cours}
		\begin{parts}
			\part[1 \half]{QCM -- les réponses sont à porter sur votre copie; il n'est pas nécessaire de justifier vos réponses; il y a une bonne réponse par question}
			\begin{subparts}
			
				\subpart Si j'ai déterminé qu'un algorithme était de complexité quadratique (également noté $\mathcal{O}(n^2)$), et si je double la taille des données en entrée, alors la durée de traitement...
				\begin{checkboxes}
					\choice ... restera la même.
					\choice ... doublera.
					\choice ... triplera.
					\correctchoice ... quadruplera.
				\end{checkboxes}
				\begin{minted}
					[
					bgcolor = gray!15,
					fontsize = \footnotesize,
					linenos = true % numéros de ligne
					]
					{Python}
def compte_elts(liste):
    compteur = 0
    for element in liste:
        compteur += 1
    return compteur
				\end{minted}
				\subpart La fonction ci-dessus a une complexité en ($n$ étant la taille de \texttt{liste})...
				\begin{checkboxes}
					\choice ... $\mathcal{O}(n^2)$.
					\correctchoice ... $\mathcal{O}(n)$.
					\choice ... $\mathcal{O}(log_2(n))$.
					\choice ... $\mathcal{O}(1)$
				\end{checkboxes}
				\subpart Un tri "stable"...
				\begin{checkboxes}
					\correctchoice ... maintient les positions relatives des éléments équivalents.
					\choice ... place les éléments en ordre décroissant.
					\choice ... a une complexité supérieure à un tri instable.
					\choice  ... a une complexité inférieure à un tri instable.
				\end{checkboxes}
				\end{subparts}
			
			\part[2] Expliquez en deux ou trois phrases le principe d'une approche gloutonne à la résolution d'un problème.
			\begin{solution}
				MCOXXXXXX: Eléments attendus:
			\end{solution}

			\part[2] Supposons que la fonction \texttt{recherche\_dicho(table, elt)} implémente la recherche dichotomique telle que nous l'avons étudiée en cours. Quelles sont les valeurs qui vont être examinées lors de l'appel \texttt{recherche\_dicho([0, 1, 1, 2, 3, 5, 8, 9, 11], 7}? \textit{Indice: la première va être table[4] qui vaut 3.}\footnote{N'hésitez pas à utiliser sur votre copie un tableau donnant les valeurs successives des indices "\texttt{debut}", "\texttt{fin}", et "\texttt{milieu}".}
			
			\begin{solution}
				MCOXXXXXX --- éléments attendus
			\end{solution}
		\end{parts}
			
		\question[6]{\textit{Détermination de complexités}}
		\newcommand{\lgcode}{0.31}
		
		\noindent
		\begin{tabular}{c c c}
			\begin{minipage}{\lgcode\linewidth}
				\begin{minted}[
					bgcolor=gray!15,
					fontsize=\footnotesize,
					linenos % Numéros de ligne
					]{python}
def f1(n):
    x = 0
    for i in range(n):
        x = x + 1
    return x
				\end{minted}
			\end{minipage}
			&
			\begin{minipage}{\lgcode\linewidth}
				\begin{minted}[
					bgcolor=gray!15,
					fontsize=\footnotesize,
					linenos % Numéros de ligne
					]{python}
def f2(n):
    x = 0
    for i in range(1000000000):
        x = x + 1
    return x
				\end{minted}
			\end{minipage}
			&
			\begin{minipage}{\lgcode\linewidth}
				\begin{minted}[
					bgcolor=gray!15,
					fontsize=\footnotesize,
					linenos % Numéros de ligne
					]{python}
def f3(n):
    x = 0
    for i in range(n):
        for j in range(5):
            x = x + 1
    return x
				\end{minted}
			\end{minipage}
			\\
			\begin{minipage}{\lgcode\linewidth}
				\begin{minted}[
					bgcolor=gray!15,
					fontsize=\footnotesize,
					linenos % Numéros de ligne
					]{python}
def f4(n):
    x = 0
    i = 0
    while i * i < n:
        x = x + 1
        i = i + 1
    return x
				\end{minted}
			\end{minipage}
			&
			\begin{minipage}{\lgcode\linewidth}
				\begin{minted}[
					bgcolor=gray!15,
					fontsize=\footnotesize,
					linenos % Numéros de ligne
					]{python}
def f5(n):
    x = 0
    for i in range(n):
        for j in range(n):
            x = x + 1
    return x
				\end{minted}
			\end{minipage}
			&
			\begin{minipage}{\lgcode\linewidth}
				\begin{minted}[
					bgcolor=gray!15,
					fontsize=\footnotesize,
					linenos % Numéros de ligne
					]{python}
def f6(n):
    x = 0
    for i in range(n):
        x = x + 1
    for j in range(n):
        x = x + 1
    return x
				\end{minted}
			\end{minipage}
		\end{tabular}
		
		Déterminer la complexité de chacune des fonctions \texttt{f1} à \texttt{f6} en fonction de $n$ (vous pouvez utiliser la notation $\mathcal{O}()$ ou la terminologie "constante / linéaire / quadratique / autre"). Il est demandé pour chaque fonction une brève phrase de justification de la réponse.

        \question[2]{\textit{Fonction de recherche de 0}}
        \begin{minted}
        	[
        	bgcolor = gray!15,
        	fontsize = \footnotesize,
        	linenos = true % numéros de ligne
        	]
        	{Python}
def compter_zeros(t):
''' Fonction qui compte le nombre de zéros après chaque élément de t'''
    n = len(t)
    compte = [0] * n # Rappel: renvoie un tableau de longueur n de 0: [0, 0, ..., 0]
    for i in range(n):
        for j in range(i+1, n):
            if t[j] == 0:
                compte[i] += 1
    return compte
    \end{minted}
       	Exécutons cette fonction sur un tableau spécifique: \texttt{compter\_zeros([1, 0, 2, 0])}. Sur votre copie, complétez le tableau suivant avec les valeurs successives des variables -- ajoutez autant de lignes qu'il y aura de passages dans la comparaison de la ligne 7 "\texttt{if t[j] == 0:}"\footnote{Indice: il y en aura 6 en tout, les deux déjà présents dans l'énoncé inclus.}; puis concluez en indiquant ce que renverra la fonction.
       	
       	\begin{tabular}{|c|c|c|c|c|c|}
       		\hline
       		\textbf{Ligne Code} & \texttt{compte} & \texttt{i} & \texttt{j} & \texttt{t[i]} & \texttt{t[j]} \\
       		\hline
       		7 & \texttt{[0,0,0,0]} & 0 & 1 & 1 & 0\\
       		\hline 
       		7 & \texttt{[1,0,0,0]} & 0 & 2 & 1 & 2\\
       		\hline 
       		7 & ... & ... & ... & ... & ...\\
       		\hline 
       	\end{tabular}
        \begin{solution}
        	MCOXXXXXX --- éléments attendus
        \end{solution}
		
		\question{\textit{Fichier CSV}} --- On considère le code suivant:
		\begin{minted}
			[
				bgcolor = gray!15,
				fontsize = \footnotesize,
				linenos = true % numéros de ligne
			]
			{Python}
import csv
fichier = open('Specialites.csv', 'r', encoding = 'utf-8')
table = list(csv.DictReader(fichier))
		\end{minted}
		Et soit le fichier \texttt{Specialites.csv} contenant les données suivantes:
		\begin{verbatim}
			Eleve,Classe,Age,Spe1,Spe2,Spe3
			Loubna,1G5,16,Maths,NSI,Physique
			Olivier,1G2,17,NSI,Maths,SES
			Lenny,1G2,17,LLC-Anglais,SES,NSI
			Anju,1G3,16,LLC-Anglais,NSI,SES
			Sophie,1G5,15,Maths,NSI,SES
		\end{verbatim}
		\begin{parts}
			\part[\half] Qu'est-ce qui va s'afficher à l'exécution du code suivant?
			\begin{minted}
				[
					bgcolor = gray!15,
					fontsize = \footnotesize,
					linenos = true % numéros de ligne
				]
				{Python}
for i in range(len(table)):
    print(table[i]['Spe2'])
			\end{minted}
			\part[1] Complétez la fonction suivante pour qu'elle fasse ce qui est spécifié.
			\begin{minted}
				[
				bgcolor = gray!15,
				fontsize = \footnotesize,
				linenos = true % numéros de ligne
				]
				{Python}
def CompteEleves(spe, table):
    ''' Fonction qui renvoie le nombre d'élèves ayant choisi la spécialité passée en argument'''
    n = len(table)
    compte = 0
    for i in range(n):
    	# A COMPLETER: le "if" qui va tester si l'élève d'indice i a la spécialité "spe"
    	...
    	    compte += 1
    return compte
			\end{minted}
			\part[2] Rédigez une fonction AgeMoy(classe) qui renvoie l'âge moyen des élèves présents dans la classe passée en argument. Par exemple \texttt{AgeMoy('1G5')} renverra la valeur $15.5$ (la moyenne de 15 et 16)\footnote{N'hésitez pas à commencer par en rédiger l'algorithme en pseudo-code: des points seront attribués à cela même si le code final est faux ou absent.}.
		\end{parts}
		
		\vspace{\baselineskip}
		\question{\textit{Ré-écriture du tri par sélection}}
		\begin{parts}
			\part[2] Ecrivez le pseudo-code d'une fonction \texttt{Prochain\_Min(liste, indice\_courant)} qui prend en entrée une liste et un indice et qui renvoie l'indice de la valeur minimale présente dans \texttt{liste} entre l'indice \texttt{indice\_courant} (inclus) et la fin de la liste. Par exemple: \texttt{Prochain\_Min([10,13,11,12], 0)} renverra $0$ (correspondant à la valeur 10) et \texttt{Prochain\_Min([10,13,11,12], 1)} renverra $2$ (correspondant à la valeur 11).
			\part[1] Traduisez le pseudo-code que vous venez de rédiger en fonction codée en Python.
			\part[2] Complétez le code suivant (parties A et B) pour qu'il réalise le tri par sélection d'une table passée en argument tel que nous l'avons vu en cours\footnote{Un petit rappel, pour gagner du temps dans la permutation: le code "\texttt{a , b = b , a}", en Python, met la valeur de a dans b et celle de b dans a.}:
				\begin{minted}
					[
						bgcolor = gray!15,
						fontsize = \footnotesize,
						linenos = true % numéros de ligne
					]
					{Python}
def TriSelect(table):
    ''' Fonction qui applique le tri par sélection à table et renvoie la table triée'''
    n = len(table)
    for i in range(n):
        # A COMPLETER - A: appel à votre fonction Prochain_Min
        ...
        # A COMPLETER - B: permutation des valeurs pour placer le ième plus petit élément à l'indice i
        ...
    return table
			\end{minted}
		\end{parts}
		
		\vspace{1.5\baselineskip} 
		\hrule
		\vspace{1.5\baselineskip} 
		
		\textit{(Question bonus 1): } En conservant le modèle que l'on a utilisé dans la question précédente pour le tri par sélection (une fonction principale et une sous-fonction qui cherche l'indice du minimum), écrivez une implémentation de ce même tri mais en ordre décroissant.
		\begin{solution}
			MCOXXXXXX --- éléments attendus
		\end{solution} 
		
		\vspace{1.5\baselineskip}
		
		\textit{(Question bonus 2): }En utilisant ce que vous avez fait à la question 6 \textit{et} ce que vous avez fait à la question bonus 1, écrivez une implémentation du tri par sélection qui classe tous les nombres pairs de la liste par ordre croissant à gauche de la liste en sortie, et tous les nombres impairs par ordre décroissant à droite. Par exemple, si on appelle cette fonction \texttt{TriSelTordu(table)}, l'appel \texttt{TriSelTordu([1, 9, 8, 10, 6, 5, 11, 23, 2])} renverra [2, 6, 8, 10, 23, 11, 9, 5, 1].
		\begin{solution}
			MCOXXXXXX --- éléments attendus
		\end{solution} 
		
		\vspace{1.5\baselineskip}
		
		\textit{(Question bonus 3): }Une technique pour repérer le plagiat dans un texte consiste à repérer les enchainements de mots (plutôt que les mots individuels). Écrivez une fonction \texttt{Plagiat(txt)} qui prend en entrée une chaîne de caractères et renvoie les 2 enchainements de mots les plus fréquents qu'elle contient. Par exemple si l'on donne à la variable txt la valeur "Le vent souffle fort sur la plaine. Les arbres dans la plaine se courbent sous le vent. Le vent, le vent, toujours le vent.", \texttt{Plagiat(txt)} renverrait les enchainements "le vent" (présent cinq fois) et "la plaine" (deux fois)\footnote{Rappel: la commande "\texttt{lst = txt.split()}" crée une liste \texttt{lst} dont les éléments sont les mots de \texttt{txt}.}.
		\begin{solution}
			MCOXXXXXX --- éléments attendus
		\end{solution} 

        \end{spacing}
    \end{questions}
\end{document}

% PAS INCLUS
		\question {\textit{Fusion de dictionnaires}}
\begin{minted}
	[
	bgcolor = gray!15,
	fontsize = \footnotesize,
	linenos = true % numéros de ligne
	]
	{Python}
	def fusionner_dictionnaires(dict1, dict2):
	resultat = dict1
	for cle, valeur in dict2.items():
	if cle in resultat:
	# A COMPLETER - A: complétez le code ici pour ajouter les valeurs des clés communes
	...
	else:
	# A COMPLETER - B: complétez le code ici pour ajouter les nouvelles clés et valeurs
	...
	return resultat
	
\end{minted}
\begin{parts}
	\part[1] Complétez (parties A et B) la fonction ci-dessus pour qu'elle renvoie un dictionnaire qui soit la fusion de \texttt{dict1} et \texttt{dict2}: toutes les clés de \texttt{dict1} et \texttt{dict2} doivent se retrouver dans le résultat et, pour les clés communes à \texttt{dict1} et \texttt{dict2}, les valeurs doivent être ajoutées. 
	\part[1] Que renverrait votre fonction ainsi complétée si on en faisait l'appel suivant: 
	
	\texttt{fusionner\_dictionnaires(\{'a': 1, 'b': 2\}, \{'b': 3, 'c': 4\})}?
\end{parts}
\begin{solution}
	MCOXXXXXX --- éléments attendus
\end{solution}
