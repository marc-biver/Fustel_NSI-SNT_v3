% !TeX document-id = {63e2cb4c-39e5-4fe5-ae31-d450d2286238}
% !TeX TXS-program:compile = txs:///pdflatex/[--shell-escape]
\documentclass[11pt,a4paper]{exam}
\addpoints % Pour compter les points
\usepackage[utf8]{inputenc}
\usepackage{minted}

\usepackage{geometry}
\usepackage{amsmath,amssymb}
\usepackage{multicol}
\usepackage{graphicx}
\usepackage{setspace}
\usepackage{dashundergaps}
\usepackage[monochrome]{xcolor} % Permet de tout mettre en N&B strict
\usepackage{xcolor}
%\selectcolormodel{gray} % Permet de tout mettre en niveaux de gris
\geometry{left=1.5cm, right=1.5cm, top=2cm, bottom=2cm} % Définition des marges du doc

%\newcommand{\class}{1\textsuperscript{ère} Spé. NSI Gr. 2\textsubscript{A}}
\newcommand{\class}{1\textsuperscript{ère} Spé. NSI Gr. 2}
\newcommand{\examnum}{Contrôle \#1}
\newcommand{\examdate}{20/10/2023}
\newcommand{\timelimit}{50 Minutes}
\newcommand{\lycee}{Lycée Fustel de Coulanges}

\pagestyle{head}
\firstpageheader{}{}{}
\runningheader{\class}{\examnum\ - Page \thepage\ / \numpages}{\examdate}
\runningheadrule


\begin{document}
% Espace d'en-tête
    \noindent
    \begin{spacing}{1}
        \noindent
        \begin{tabular*}{\textwidth}{l @{\extracolsep{\fill}} l @{\extracolsep{6pt}} l}
            \textbf{\class} & \textbf{\examnum, \examdate}&\\
            \textbf{\lycee} &\textbf{Durée: \timelimit} &\\
        \end{tabular*}\\
    \end{spacing}

    \noindent
    \vspace{10pt}
    \hrule
    \vspace{5pt} 
    \noindent
    \\
    Ce contrôle comporte \numquestions\ questions; il sera noté sur \numpoints\ points. 
    Les réponses sont à porter sur une copie \uline{comportant votre nom}.
    Il n'est pas nécessaire de répondre aux questions dans l'ordre \textemdash\ commencez par celles où vous vous sentez le plus à l'aise \textit{(mais ne tentez les questions bonus qu'après avoir fini le reste!!)}.\\
    Une attention particulière sera portée à la qualité et à la clarté du code que vous aurez écrit.\\
    \noindent
    \hrule
    \vspace{15pt} 

    \begin{questions} % DEBUT DE L'EXAMEN
        \begin{spacing}{1}
    
           \question Questions à choix multiples (aucune justification de la réponse n'est nécessaire):
   		        \begin{parts}
       			\part[1] Quelle est la syntaxe correcte pour un '\texttt{if}' en Python?
       			\frenchspacing
%       			\emph{frenchspacing} % pb des espaces après deux points       			
   			        \begin{choices}
         				\choice \texttt{if condition ( ... )}
						\choice \texttt{if condition: ...}
         				\choice \texttt{if: condition \{ ... \} }
         				\choice \texttt{if condition \{ ... \}}
           			\end{choices}
           		\nonfrenchspacing
       			\part[1] Si on souhaite répéter un bloc de code tant qu'une condition est vraie, quelle structure utilise-t-on en Python?
					\frenchspacing
					\begin{choices}
						\choice \texttt{do \{ ... \} while condition}
						\choice \texttt{repeat ... until condition}
						\choice \texttt{while condition: ...}
						\choice \texttt{for condition: ...}
					\end{choices}
					\nonfrenchspacing
				\part[1] Comment ajoute-t-on un élément à la fin d'une liste en Python?
					\frenchspacing
					\begin{choices}
						\choice \texttt{liste.add(element)}
						\choice \texttt{liste.insert(element)}
						\choice \texttt{liste.append(element)}
						\choice \texttt{liste.push(element)}
					\end{choices}
				\part[1] Quelle est la sortie du code suivant?
					\begin{minted}
						[
						bgcolor = gray!15,
						fontsize = \footnotesize,
						linenos = true % numéros de ligne
						]
						{Python}
txt = "ha"
for i in range(3):
    txt = txt + "ha"
print(txt)
				\end{minted}
					\begin{choices}
						\choice \texttt{ha}
						\choice \texttt{haha}
						\choice \texttt{hahaha}
						\choice \texttt{hahahaha}
					\end{choices}
					\nonfrenchspacing
	           	\end{parts}
							
            \question {Fonction d'échange d'éléments dans une liste} 
            \begin{parts}
            	\part[2] {Ecriture du code:} Complétez la fonction '\texttt{Echange(L, i, j)}' ci-dessous qui prend en entrée une liste \texttt{L}, deux indices \texttt{i} et \texttt{j}, et retourne une liste où les éléments \texttt{L[i]} et \texttt{L[j]} auront été échangés et le reste de la liste sera identique à \texttt{L}.
            	\noindent
            	\begin{minted}
            		[
            		bgcolor = gray!15,
            		fontsize = \footnotesize,
            		linenos = true % numéros de ligne
            		]
            		{Python}
def Echange(L, i, j):
    ## A COMPLETER: code effectuant l'echange des elements d'indice i et j
    return ...
            	\end{minted}
            	\part{Exécution du code:}
            	\begin{subparts}
            		\subpart[\half] Que retournera l'appel \texttt{echange([1,2,3], 1, 2)}?
            		\subpart[\half] Que retournera l'appel \texttt{echange([1,2,3], 2, 3)}?
            	\end{subparts}
            \end{parts}
            
            
		\newpage
          \question[2] \textit{Comptage jusqu'à 10:} Complétez le code ci-dessous pour afficher à l'écran les nombres de 1 à 10 \textbf{en utilisant une boucle "while"}.
			\noindent
			\begin{minted}
			[
			bgcolor = gray!15,
			fontsize = \footnotesize,
			linenos = true % numéros de ligne
			]
			{Python}
def compte10():
    i = 0
    while ...
        ## A COMPLETER: code affichant, au moyen d'une boucle while, les nombres de 1 a 10
        print(i)
			\end{minted}          
            
            \question[1] \textit{Recherche dans une liste:} Ecrivez une fonction '\texttt{recherche\_element(liste, element)}' qui prend une liste et un élément en entrée, et renvoie '\texttt{OUI}' si l'élément est présent dans la liste, sinon renvoie '\texttt{NON}'.
            
           \question[3] \textit{Evolution du nombre d'abonnés d'une chaîne YouTube:} Un vlogueur commence avec 1000 abonnés sur sa chaîne YouTube. Chaque mois, il gagne 3\% de nouveaux abonnés en plus et il perd 50 abonnés qui se désinscrivent. Écrivez une fonction '\texttt{NbAbos(nbMois)}' qui retourne une liste contenant l'évolution du nombre d'abonnés sur un nombre \texttt{nbMois} de mois. \textit{Indice: pour conserver la partie entière d'un nombre \texttt{N} plus 3\%, on utilise la formule \texttt{int(1.03 * N)}}.
            
            \question[2] \textit{Table de multiplication:} Complétez la fonction '\texttt{Multiples(num)}' ci-dessous: elle prend en entrée un nombre entier '\texttt{num}' et retourne la liste de ses dix premiers multiples (donc \texttt{[num*1, num*2, num*3 .... num*10]}).
            \begin{minted}
            	[
            	bgcolor = gray!15,
            	fontsize = \footnotesize,
            	linenos = true % numéros de ligne
            	]
            	{Python}
def Multiples(num):
Lst ....
for i ...
    ## A COMPLETER: code constituant la liste des multiples de num
return Lst
            \end{minted}
            
            \question[2] \textit{Vérification de parité:} Ecrivez une fonction '\texttt{est\_pair(n)}' qui prend un entier '\texttt{n}' en entrée et renvoie '\texttt{OUI}' si '\texttt{n}' est pair, sinon renvoie '\texttt{NON}'.
            
            \noindent
            \question[3] Correction de code: même principe que dans le l'interrogation précédente \textemdash\ dans les quatre extraits ci-dessous, il y a trois erreurs à identifier et à corriger.
            \\
            \begin{minipage}{0.5\textwidth}
            	\begin{minted}
            		[
            		bgcolor = gray!15,
            		fontsize = \footnotesize,
            		linenos = true % numéros de ligne
            		]
            		{Python}
# A: Affichage des carres d'entiers 0 a 5
for i in range(6):
    print(i ** 2)

# B: Compte a rebours partant de 10
x = 10
while x > 0:
print(x)
x = x - 1
            	\end{minted}
            \end{minipage}
            \begin{minipage}{0.5\textwidth}
            	\begin{minted}
            		[
					bgcolor = gray!15,
					fontsize = \footnotesize,
					linenos = true % numéros de ligne
					]
					{Python}
# C: Affichage des 10 premiers entiers
i = 0
while i < 10:
    print(i)

# D: Doublerment dernier element d'une liste
def DoubleDernier(Lst):
    a = len(Lst)
    resultat = Lst[a] * 2
    return resultat
            	\end{minted}
            \end{minipage}
            
            \vspace{10pt} 
            \hrule
            \vspace{15pt} 
            
            \textit{(Question bonus 1): }Écrivez un programme qui demande à l'utilisateur deux nombres entiers '\texttt{a}' et '\texttt{b}', puis utilise une boucle \texttt{while} pour calculer le produit '\texttt{a x b}' sans utiliser l'opérateur de multiplication (\texttt{*}).
            
            \vspace{15pt}

	        \textit{(Question bonus 2): }Écrivez un programme qui demande à l'utilisateur d'entrer une liste de nombres (la fin de la liste est marquée par l'entrée d'un nombre négatif), puis utilise une boucle \texttt{for} pour calculer et afficher la moyenne de ces nombres.
            
        \end{spacing}
    \end{questions}
\end{document}